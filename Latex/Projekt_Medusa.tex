% Präambel
\documentclass[12pt,a4paper,oneside, 
liststotoc, 					% Tabellen- und Abbildungsverzeichnis ins Inhaltsverzeichnis
bibtotoc,						% Literaturverzeichnis ins Inhaltsverzeichnis aufnehmen
titlepage, 						% Titlepage-Umgebung statt \maketitle
headsepline, 					% horizontale Linie unter Kolumnentitel
%abstracton,					% Überschrift beim Abstract einschalten, Abstract muss dazu in {abstract}-Umgebung stehen
%DIV11,							% auskommentieren, um den Seitenspiegel zu vergrößern
BCOR6mm,						% Bindekorrektur, die den Seitenspiegel um 6mm nach rechts verschiebt,
]{scrreprt}		

% Laden verschiedener Packages
\usepackage{ucs} 				% Dokument in utf8-Codierung schreiben und speichern
\usepackage{url}
\usepackage[utf8]{inputenc} 	% ermöglicht die direkte Eingabe von Umlauten
\usepackage[english, ngerman]{babel} 	% deutsche Trennungsregeln und Übersetzung der festcodierten Überschriften
\usepackage[T1]{fontenc} 		% Ausgabe aller zeichen in einer T1-Codierung (wichtig für die Ausgabe von Umlauten!)
\usepackage{graphicx}  			% Einbinden von Grafiken erlauben
\usepackage{amsmath}
\usepackage{csquotes}           % Zusatzpaket für Zitierungen (Nur optisch, nicht logisch)
%\usepackage{amsfonts}
%\usepackage{amssymb}
\usepackage{mathpazo} 			% Einstellung der verwendeten Schriftarten
\usepackage{textcomp} 			% zum Einsatz von Eurozeichen u. a. Symbolen
\usepackage{listings}			% Datstellung von Quellcode mit den Umgebungen {lstlisting}, \lstinline und \lstinputlisting
\usepackage[dvipsnames]{xcolor} % einfache Verwendung von Farben in nahezu allen Farbmodellen
\usepackage[intoc]{nomencl} 	% zur Erstellung des Abkürzungsberzeichnisses
\usepackage{fancyhdr}			% Zusatzpaket zur Gestaltung von Fuß und Kopfzeilen
\usepackage{epigraph}
\usepackage{svg}
\usepackage{tkz-kiviat}

\usepackage{xargs}
\usepackage{setspace}
\usepackage[smaller]{acronym}

\usepackage[]{algorithm2e} 		% Für die Verwendung von Pseudocode
\usepackage[]{algorithmicx}		% Anderes Pseudocode Paket zum testen

\usepackage[colorinlistoftodos,prependcaption,textsize=tiny]{todonotes}

\usepackage{hyperref}

%Einstellungen, Farben, Commands, 
\newcommandx{\unsure}[2][1=]{\todo[linecolor=red,backgroundcolor=red!25,bordercolor=red,#1]{#2}}
\newcommandx{\change}[2][1=]{\todo[linecolor=blue,backgroundcolor=blue!25,bordercolor=blue,#1]{#2}}
\newcommandx{\info}[2][1=]{\todo[linecolor=OliveGreen,backgroundcolor=OliveGreen!25,bordercolor=OliveGreen,#1]{#2}}
\newcommandx{\improvement}[2][1=]{\todo[linecolor=Plum,backgroundcolor=Plum!25,bordercolor=Plum,#1]{#2}}
\newcommandx{\thiswillnotshow}[2][1=]{\todo[disable,#1]{#2}}

% Default fixed font does not support bold face
\DeclareFixedFont{\ttb}{T1}{txtt}{bx}{n}{12} % for bold
\DeclareFixedFont{\ttm}{T1}{txtt}{m}{n}{12}  % for normal

%------------Colors ----------------
\definecolor{deepblue}{rgb}{0,0,0.5}
\definecolor{deepred}{rgb}{0.6,0,0}
\definecolor{deepgreen}{rgb}{0,0.5,0}

\pgfkeys{
	/kiviatgrad/simplify label/.code={
		\ifx\nv\undefined\else
		\pgfmathparse{Mod(\nv,5)}
		\ifdim\pgfmathresult pt>0pt
		\tikzset{opacity=0}
		\fi
		\fi
	}
}	
% ------------------Setting up environment for codes ---------------
% Python style for highlighting
\newcommand\pythonstyle{\lstset{
		language=Python,
		numbers=left,			% Zeilennummern links
		stepnumber=1,			% Jede Zeile nummerieren.
		numbersep=5pt,			% 5pt Abstand zum Quellcode
		breaklines=true,		% Zeilen umbrechen wenn notwendig.
		breakautoindent=true,	% Nach dem Zeilenumbruch Zeile einrücken.
		postbreak=\space,		% Bei Leerzeichen umbrechen.
		tabsize=2,				% Tabulatorgrösse 2
		basicstyle=\ttfamily\footnotesize, % Nichtproportionale Schrift, klein für den Quellcode
		showspaces=false,		% Leerzeichen nicht anzeigen.
		showstringspaces=false,	% Leerzeichen auch in Strings ('') nicht anzeigen.
		extendedchars=true,		% Alle Zeichen vom Latin1 Zeichensatz anzeigen.
		captionpos=b,			% sets the caption-position to bottom
		xleftmargin=0pt,		% Rand links
		xrightmargin=0pt,		% Rand rechts
		frame=single,			% Rahmen an
		frameround=ffff,
		rulecolor=\color{darkgray},	% Rahmenfarbe
		basicstyle=\ttm,
		morekeywords={access,and,break,class,continue,def,del,elif,else,except,exec,finally,for,from,global,if,import,in,is,lambda,not,or,pass,print,raise,return,try,while},
		morekeywords=[2]{abs,all,any,basestring,bin,bool,bytearray,callable,chr,classmethod,cmp,compile,complex,delattr,dict,dir,divmod,enumerate,eval,execfile,file,filter,float,format,frozenset,getattr,globals,hasattr,hash,help,hex,id,input,int,isinstance,issubclass,iter,len,list,locals,long,map,max,memoryview,min,next,object,oct,open,ord,pow,property,range,raw_input,reduce,reload,repr,reversed,round,set,setattr,slice,sorted,staticmethod,str,sum,super,tuple,type,unichr,unicode,vars,xrange,zip,apply,buffer,coerce,intern},%
		morecomment=[l]\#,%
		sensitive=true,%
		%otherkeywords={self,if,print,while,return,def},             % Add keywords here
		keywordstyle=\ttb\color{deepblue},
		emph={MyClass,__init__},          % Custom highlighting
		emphstyle=\ttb\color{deepred},    % Custom highlighting style
		stringstyle=\color{deepgreen},
		% Any extra options here
		showstringspaces=false            % 
}}

% Python environment
\lstnewenvironment{python}[1][]
{
	\pythonstyle
	\lstset{#1}
}
{}

% Python for external files
\newcommand\pythonexternal[2][]{{
		\pythonstyle
		\lstinputlisting[#1]{#2}}}


\usetikzlibrary{arrows} % define style of tkiz kiviat

% Abkürzungen
\newcommand{\ua}{\mbox{u.\,a.\ }}
\newcommand{\zB}{\mbox{z.\,B.\ }}
\newcommand{\bs}{$\backslash$}

\renewcommand{\nomname}{Abkürzungsverzeichnis}

% -------------------------------------------------------------------------------------------
% Definition der Kopf- und Fußzeilen
\lhead{}								% Kopf links
\chead{}								% Kopf mitte
\rhead{\sffamily{\leftmark}}				% Kopf rechts
\cfoot{}								% Fuß links
\lfoot{\sffamily{Seite \thepage}}				% Fuß mitte
\rfoot{\sffamily{\autorShort}}				% Fuß rechts
\renewcommand{\headrulewidth}{0.4pt}	% Liniendicke Kopf
\renewcommand{\footrulewidth}{0.4pt}	% Liniendicke Fuß


\makenomenclature							% Abkürzungsverzeichnis erstellen
%\input{Inhalt/abkuerzungen}					% Datei mit Abkürzungen laden

% Definition of colors
\definecolor{lightblue}{rgb}{0.910,0.933,0.970}
\definecolor{lightred}{RGB}{247,238,232}
\definecolor{monochromeLightblue}{RGB}{165,188,222}
\definecolor{monochromeLightred}{RGB}{222,188,165}
\definecolor{kiviatOne}{RGB}{137,193,30}
\definecolor{kiviatTwo}{RGB}{20,128,120}
\definecolor{kiviatThree}{RGB}{208,108,32}
\definecolor{kiviatFour}{RGB}{168,26,104}

\newcommand\ColorBox[1]{\textcolor{#1}{\rule{2.5ex}{2.5ex}}}

% ----------------------------------- Links Styling -----------------------------------------
\hypersetup{
	%pdfborder = {0 0 0},
	bookmarks=true,         % show bookmarks bar?
	unicode=false,          % non-Latin characters in Acrobat’s bookmarks
	pdftitle={\titel}
	pdfauthor={\autor},     % author
	pdfsubject={\ausarbeitung},   % subject of the document
	pdfcreator={\autor},   % creator of the document
	pdfproducer={\autor}, % producer of the document
	colorlinks=true,       % false: boxed links; true: colored links
	linkcolor=blue,          % color of internal links (change box color with inkbordercolor)
	citecolor=CadetBlue,        % color of links to bibliography
	filecolor=magenta,      % color of file links
	urlcolor=cyan,           % color of external links
}

% -------------------------------------------------------------------------------------------
%                     Persönliche Daten
% -------------------------------------------------------------------------------------------


\newcommand{\titel}{Erstellung von Irrbildern zur Überlistung einer verkehrsschilderkennenden KI}
\newcommand{\shellcmd}[1]{\\\indent\indent\texttt{\footnotesize\# #1}\\}
\newcommand{\untertitel}{}
\newcommand{\arbeit}{Projekt Bericht}
\newcommand{\studiengang}{Informatik}
\newcommand{\autor}{Leonhard Applis, Peter Bauer,Florian Stöckl, Andreas Porada}
\newcommand{\matrikelnr}{2086307}
\newcommand{\abgabe}{24.09.2018}
\newcommand{\betreuerth}{Prof. Dr. Gallwitz}
\newcommand{\jahr}{2018}			% für Angabe im Copyright-Vermerk der Titelseite


% -------------------------------------------------------------------------------------------
%                     Beginn des Dokumenteninhalts
% -------------------------------------------------------------------------------------------


\begin{document}
\setcounter{secnumdepth}{3}					% Nummerierungstiefe fürs Inhaltsverzeichnis
\setcounter{tocdepth}{3}
\sffamily									% für die Titelei serifenlose Schrift verwenden

% ------------------------------ Titelei -----------------------------------------------------

\thispagestyle{plain}
\begin{titlepage}
\enlargethispage{3.5cm}
\sffamily 								% Serifenlose Grundschrift für die Titelseite einstellen
\begin{minipage}{\textwidth}
	\vspace{-2cm}
	\noindent \includegraphics[scale=0.2]{Images/Logo_TH.png} \hfill
\end{minipage} 
\begin{center}

\huge{\textsc{\textbf{\titel}}}\\[1.5ex]
\Large{\textbf{\untertitel}}\\[5ex]
\LARGE{\textbf{\ausarbeitung}}\\[2ex]
\normalsize{~}\\[3ex]
\Large{Master-Studiengang \textit{\studiengang}}\\[1ex]
\normalsize{Technische Hochschule Georg Simon Ohm}\\[5ex]
von\\[1ex] \autor \\[12ex]

\begin{tabular}{ll}
	Abgabedatum:					& \quad \abgabeGI \\ 
	
	%einkommentieren für TH Abgabe
	%Abgabedatum:					& \quad \abgabeTH \\ 
	
	%einkommentieren für TH Abgabe
	%Gutachter der Hochschule: & \quad \betreuerth \\ 
	%[6ex]%formerly 5ex
	
\end{tabular} 

\end{center}

\end{titlepage}
%Starting Correct Spacing here...
\onehalfspacing 				% erzeugt die Titelseite
\pagenumbering{Roman}						% große, römische Seitenzahlen für Titelei
\addchap*{Eidesstattliche Erklärung}
Wir versichern hiermit, dass der \arbeit~ mit dem Thema

\emph{\titel}\\
selbständig verfasst und keine anderen als die angegebenen Quellen und Hilfsmittel benutzt habe. Die Arbeit wurde bisher keiner anderen Prüfungsbehörde vorgelegt und auch nicht veröffentlicht.


Wir versichern zudem, dass die eingereichte elektronische Fassung mit der gedruckten Fassung übereinstimmt.\\[10ex]

Nürnberg, den \today \\[4ex]


\rule[-0.2cm]{10cm}{0.5pt} \\

\textsc{\autor} \\[10ex]
 				% Einbinden der eidestattlichen Erklärung
\chapter*{Abstract} %*-Variante sorgt dafür, das Abstract nicht im Inhaltsverzeichnis auftaucht
This paper uses several scientific approaches to show how convolutional neural networks can be tricked into recognizing and correctly classifying road signs.
In the field of autonomous driving, neural networks have drastically increased the recognition rate, but they are still susceptible to errors in the face of deliberately generated false images. Even without information about the underlying architecture (so-called black box attacks), attacks by otherwise generated false images should be possible.
The presented methods \textit{Degeneration}, \textit{Saliency Maps} and \textit{Gradient Ascent} are successfully applied to generate false images for attacks on an unknown neural network with the help of an own neural network which serves as \textit{Substitute}.
An attack is considered "'successful"' if the image is recognized as a street sign with a confidence greater than 90\%, which a human observer would not recognize as such.
~\newline
~\newline
\begin{flushleft}
	\begin{tabular}{lp{11cm}}
		\textbf{title:} & Fooling a TrafficSign-AI \\
		\textbf{authors:}  & \autor \\
		
		%einkommentieren für TH Abgabe
		%\textbf{reviewer TH:} & \betreuerth \\
		%[6ex]%formerly 5ex
	\end{tabular} 
\end{flushleft}


\chapter*{Kurzfassung} 
Diese Arbeit zeigt anhand von mehreren wissenschaftlichen Ansätzen, wie Convolutional Neural Networks zur Erkennung und Klassifikation von Straßenschildern überlistet werden können.
Im Bereich des Autonomen Fahren wurde mit Neuronalen Netzen die Erkennungsrate drastisch gesteigert, allerdings sind diese immer noch fehleranfällig gegenüber gezielt erzeugten Irrbildern. Selbst ohne Informationen über die unterliegende Architektur (sog. Black Box Angriffe), sind Angriffe durch anderweitig erzeugte Irrbilder möglich.
Die vorgestellten Verfahren \textit{Degeneration}, \textit{Saliency Maps} und \textit{Gradient Ascent} werden erfolgreich angewendet, um mithilfe eines eigenen Neuronalen Netz, welches als \textit{Substitute} dient, Irrbilder für Angriffe auf ein unbekanntes Neuronales Netz zu erzeugen.
Ein Angriff gilt als "'erfolgreich", wenn das Bild mit einer Konfidenz größer als 90\% als Straßenschild erkannt wird, welches ein menschlicher Betrachter nicht als solches erkennen würde.

~\newline
\begin{flushleft}
	\begin{tabular}{lp{11cm}}
		Titel:&  \titel \\ 
		Authoren:&  \autor \\
		%einkommentieren für TH Abgabe
		%Prüfer der Hochschule: &  \betreuerth \\ 
		%[6ex]%formerly 5ex	
	\end{tabular} 
\end{flushleft}
   				% Einbinden des Abstracts

\tableofcontents							% Erzeugen des Inhalsverzeichnisses
\printnomenclature[2.0cm]					% Erzeugen des Abkürzungsverzeichnisses
\listoffigures 								% Erzeugen des Abbildungsverzeichnisses 
%\listoftables 								% Erzeugen des Tabellenverzeichnisses
\pagebreak
% ------------------ Graphic Extension ------------------------------------------------------

% --------------------------------------------------------------------------------------------
%                    Inhalt der Bachelorarbeit
%---------------------------------------------------------------------------------------------
\pagenumbering{arabic}						% arabische Seitenzahlen für den Hauptteil
\pagestyle{fancy}					
\rmfamily

%% alle Abkürzungen, die in der Arbeit verwendet werden
\chapter*{Abkürzungsverzeichnis}
\addcontentsline{toc}{chapter}{Abkürzungsverzeichnis}
\begin{singlespace}
	
\begin{acronym}[Studienarbeit]
    \acro{DBMS}{Database Management System}
	\acro{DHBW}{Duale Hochschule Baden-Württemberg}
	\acro{SQL}{Structured Query Language}
	\acro{ETL}{Extract-Transform-Load}
\end{acronym}

\end{singlespace}
\chapter{Einleitung}
\label{cha:Einleitung}
\setlength{\epigraphwidth}{4in}

\section{Ziel der Arbeit}
\label{sec:ZielDerArbeit}

\section{Aufbau der Arbeit}

\section{Verwandte Werke und Primärquellen}
\label{sec:VerwandteWerke}

\section{Rahmenbedingungen des Informaticups}

\chapter{Ansatz 1 : Greyboxing [LeFl]}
\label{Cha:GreyBoxing}
\chapter{Ansatz 2 : Genetische Algorithmen [AnPe]}
\label{GenAlgo}
\chapter{Fazit}
\label{cha:Fazit} \label{cha:Schluss}
Die vorangehende Arbeit stellt verschiedene bilderzeugende Verfahren zum Angriff einer Verkehrsschilder erkennenden \ac{KI} vor.\\
Die Motivation für die Arbeit ist das Forschungsfeld des autonomen Fahrens, in welchem \acp{KI} für die optische Erkennung der Umwelt, im besonderen für das Erkennen von Straßenschildern, verwendet werden sollen. Vorangehende Arbeiten mit Neuronalen Netzen zeigten, dass diese zum aktuellen Zeitpunkt anfällig für Manipulationen sind und fehlerhaft reagieren können. Das gezielte Ausnutzen dieser Fehleranfälligkeit wird in der Forschung als \textit{Adversarial Attack} bezeichnet. Das Ziel der Arbeit ist es verschiedene Verfahren vorzustellen und zu untersuchen, mit denen \textit{Adversarial Attacks} gegen ein \ac{NN} zur Erkennung von Straßenschildern durchgeführt werden können. Die Aufgabenstellung und das \ac{NN} werden von der \acl{GI} im Rahmen des Wettbewerbs \textit{InformatiCup 2019} gestellt.

Im ersten Schritt der Arbeit werden die Anforderungen an die bilderzeugenden Verfahren zum Angriff der \ac{KI} des \ac{GI}-Wettbewerbs analysiert und die Rahmenbedingungen ermittelt. Die Hauptquelle dafür ist die Aufgabenstellung des InformatiCups 2019, sowie die zur Verfügung stehende \ac{KI} des Wettbewerbs, welche über eine Webschnittstelle erreichbar ist.

In Kapitel \ref{cha:TechKonzept} werden die verwendeten Technologien der Arbeit vorgestellt. Diese dienen als Grundlage für die verschiedenen Verfahren in den Kapiteln \ref{cha:Degeneration} - \ref{cha:gascent}. Anschließend wird erläutert, warum sich \textit{Adversarial Attacks} für die Zielerreichung der Arbeit eignen. Der letzte Abschnitt des Kapitels behandelt die Implementierung eines eigenen Modells zur Klassifizierung von Straßenschildern, welches im Projekt \textit{Aphrodite} genannt wird. Das Modell wird in dem Verfahren lokale Degeneration in Kapitel \ref{cha:Degeneration} und im \textit{Saliency Maps} Verfahren in Kapitel \ref{cha:saliency} eingesetzt. Das eigene Modell dient als Ersatz für das \ac{NN} des \ac{GI}-Wettbewerbs, da Anfragen an letzteres Beschränkungen wie einer Anzahl an Requests pro Minute unterliegen. Zusätzlich bietet ein lokales Modell Möglichkeiten zur detaillierten Untersuchung der Verfahren, da es sich nicht um ein Black Box Modell handelt.
 
Das erste vorgestellte bilderzeugende Verfahren zum Angriff einer Verkehrsschilder erkennenden \ac{KI} ist die Degeneration. Bei diesem Verfahren wird iterativ ein Bild manipuliert, welches ein Objekt enthält, dass vom Menschen und dem \ac{NN} des \ac{GI}-Wettbewerbs als Straßenschild klassifiziert wird. Durch verschiedene Bildbearbeitungsfunktionen wird das Bild solange verändert, bis ein Mensch kein Straßenschild mehr erkennt. Bei jeder Iteration wird darauf geachtet, dass das veränderte Bild von der \ac{KI} weiterhin mit hoher Konfidenz einer Straßenschild-Klasse zugeordnet wird. Die Degeneration ist ein vergleichsweise simpler Algorithmus, welcher zuverlässig gute Ergebnisse liefert.\\ 
Der wesentliche Vorteil der Degeneration ist die Unabhängigkeit vom Modell, welches angegriffen werden soll. Es muss kein Transfermodell erstellt werden, um angepasste Angriffe zu erzeugen, sondern der Algorithmus erzeugt Angriffsbilder anhand der Antworten der Black Box \ac{KI} des \ac{GI}-Wettbewerbs.\\
Diese Tatsache stellt gleichzeitig einen Nachteil der Degeneration dar, da Anfragen an die Webschnittstelle des \ac{GI}-Wettbewerbs mehr Zeit benötigen als Anfragen an ein lokales \ac{NN}. Für einen Angriff, bei dem ein hoher Konfidenzwert für eine Straßenschild-Klasse erreicht werden soll, muss über eine Stunde gerechnet werden. Die Zeit erhöht sich weiter, wenn das angegriffene \ac{NN} dynamisch auf Rauschen reagiert und zugesendete Bilder für ein kontinuierliches Training nutzt. Da eine Grundvoraussetzung des Wettbewerbs jedoch ein abgeschlossenes trainiertes Netz ist, gehört dieses Problem nicht zum Rahmen der Arbeit. \\
Trotz des Zeitaufwands ist die Effektivität der Degeneration vergleichsweise hoch, obwohl es sich im Wesentlichen um einen Brute-Force Angriff handelt. 

~\newline Des Weiteren konnten mit den optimierten \textit{Saliency Map} Verfahren Erfolge in den geglätteten Varianten verbucht werden. Somit konnte bestätigt werden, dass die Visualisierung der relevanten Pixel für ein Bild mit hoher Konfidenz geeignet sind, um als \textit{minimale Beispiele} für das NN verwendet werden können. 
Die entsprechenden Verfahren erzeugten Täuschungen mit Konfidenzen >0.9, allerdings lassen sich keine Aussagen über die allgemeine Verlässlichkeit und Zielgerichtetheit treffen.

Zuletzt können auch mit dem \textit{Gradient Ascent} Verfahren sehr gute Ergebnisse erzielt werden. Für 10 von 43 Klassen können Täuschungen mit hohen Konfidenzen am \ac{NN} des \ac{GI}-Wettbewerbs erzielt werden. Es wird vermutet, dass die Ausbeute mit weiterer Optimierung vergrößert werden kann oder auch echte Bilder für die Erzeugung der Täuschungen verwendet werden können. 


\section{Diskussion}
Die Herausforderungen des Wettbewerbs wirken sich auf die Erfolge der Verfahren aus. Es wird vermutet das beide Verfahren \textit{Saliency Map} und \textit{Gradient Ascent} bessere Ergebnisse liefern könnten, wenn das verwendete Bild größer als $64\times64$ wäre. Des Weiteren kann nichts über die Validierungsgenauigkeit des \ac{NN} des Wettbewerbs gesagt werden, weshalb auch die nicht zielgerichteten Täuschungsbilder als gutes Ergebnis betitelt werden. 


Beim Vergleich des \textit{Saliency Map} und \textit{Gradient Ascent} Verfahrens kann das zuletzt genannte bevorzugt werden. 

~\newline Diese Methoden lassen sich schwer mit der Degeneration vergleichen - die schnelleren Erfolge werden voraussichtlich mithilfe der Degeneration erzeugt, da sie deutlich weniger Vorlauf benötigt.
Nachdem allerdings das entsprechende Umfeld (eigenes Netz, Bibliotheken und Code) erzeugt wurde, können innerhalb des \textit{Gradient Ascent} Foolings deutlich schneller zuverlässige Bilder erzeugt werden.  
%content 
%vergleich der ergebnisse (reproduzierbarkeit, gezielt/random, qualität der bilder) - einschränkungen probleme unserer ansätze, wichtig: eigene fehler oder einschränkungen der methodik erkennen.



\section{Weiterführende Arbeiten}~\newline 
Die Ergebnisse dieser Arbeit liefern weitere Ansätze für zukünftige Aufgaben. Zum einen können die verwendeten Ansätze individuell weiter optimiert werden, bezüglich des selbsterstellten lokalen \acl{NN} und der Algorithmik bzw. deren Parameter. Zum anderen können Verfahren entwickelt werden, welche sich aller Methoden gezielt bedienen. 


~\newline Eine insgesamt spannende Arbeit wäre die Anwendung von Adversarial Attacks auf Sprachassistenten. 
Vor allem die Degeneration kann bereits in ihrem jetzigen Zustand genutzt werden, um Störgeräusche zu erzeugen, welche dennoch als Schlüsselwörter erkannt werden und ein \textit{Smarthome} hacken. 

Auch die anderen Verfahren sind insbesondere geeignet, sollte das Modell offenliegen. Der Sprachassistent-Hersteller Mycroft setzt auf Open-Source und stellt dementsprechend auch das (allgemeine) Model bereit. 
Zu bemerken ist hierbei noch, dass der Nutzer innerhalb der ersten Aktionen den Sprachassistenten auf seine Aussprache konfiguriert.  

~\newline An der Degeneration können ebenfalls großflächige Weiterentwicklungen vorgenommen werden: 

Zum einen die Verwendung der Tree-Degeneration, zum anderen können Beschleunigung und Verfall der einzelnen Alternations eingebaut werden. Ebenso sollte ein kleines Script erstellt werden, welche die verschiedenen Manipulationsfunktionen kurz auslotet und dem Nutzer vorstellt. 

Als komplexere Weiterentwicklung können mithilfe der Degeneration \textit{Manipulationsvektoren} erstellt werden und in einem Raum abgebildet werden. Hierbei stellt jedes \textit{Rauschen} (bzw. Bilddifferenz) und die Score-Differenz einen Vektor dar, welcher in einem Raum abgebildet werden kann.   

Mithilfe dieser Vektoren könnte über statistische bzw. numerische Verfahren solch ein Vektor gefunden werden, welcher die größte Länge hat allerdings die geringste Score-Differenz aufweist. Angewandt auf das Urbild sollte dieser Vektor ein optimales Ergebnis erzielen.  

~\newline Einen weiteren Blick sollte man der Überführbarkeit der Angriffe von einem lokalen Model auf ein unbekanntes Model widmen, wie es in Abschnitt \ref{sec:TrasiModell} thematisiert wird: 

Diese Eigenschaft werden von den Saliency-Maps und dem \textit{Gradient Ascent} Fooling hinreichend erfüllt, wohingegen solche Versuche beim Degeneration-Ver"-fahren scheitern. 

Im Rahmen dieses Ergebnisses könnte man einen gezielten Test der Verfahren auf zwei bekannte Modelle durchführen, um hier transparentere Werte zu erhalten und Gründe für dieses Verhalten auszumachen. 
%content
%Vergleich mit einem Framework, wie zum Beispiel CleverHans \footnote, Vergleich von einem gut und schlecht trainierten Modell, wie arg unterscheiden sich daraus generierte Bilder (lohnt es sich), wie wichtig ist die Bildgröße, 
\newpage\newpage

% ---------------------------- Literaturverzeichnis ----------------------------------------------

\bibliographystyle{plain}
\bibliography{src}


% ------------------------------- Anhang ---------------------------------------------------------

\begin{appendix}
\clearpage
\pagenumbering{Roman}						% römische Seitenzahlen für Anhang
\end{appendix}


\end{document}
