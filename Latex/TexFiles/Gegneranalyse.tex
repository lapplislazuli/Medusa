\chapter{Anforderungsanalyse und Rahmenbedingungen}
\label{cha:AnfAnalyse}

Im nachfolgenden Kapitel werden ausgehend von der Aufgabenstellung des InformatiCups 2019 Anforderungen an die Bilderzeugungsverfahren zur Überlistung einer Verkehrsschilder erkennenden \ac{KI} analysiert. Die Anforderungen untergliedern sich dabei in funktionale Anforderungen und nichtfunktionale Anforderungen. Die analysierten Anforderungen werden durch die Rahmenbedingungen des Wettbewerbs ergänzt, die im wesentlichen aus dem bereitgestellten \acl{NN} besteht.

\section{Funktionale Anforderungen}
Die Kernaufgabe der implementierten Lösungen der nachfolgenden Kapitel \ref{cha:Degeneration} bis \ref{cha:gascent} besteht darin Bilder zu genieren. Als Eingangsparameter dürfen dafür Bilder verwendet werden, die anschließend durch Algorithmen modifiziert werden.

Die implementieren Lösungen müssen nachvollziehbar und reproduzierbar sein. Das bedeutet, dass eine Modifikation mit den selben Eingangsparametern zu dem selben Ergebnis führen muss. Grundvoraussetzung für diese Anforderung 

\section{Nichtfunktionale Anforderungen}

%Content
%Bewertung nach Confidence und nach Spektrum der Klassen des GTSRB Datensatzes


\section{Eigenschaften des \acl{NN}es des GI Wettbewerbs}
\label{sec:EigenschaftenTrasi}




33 verschiedene aufgezeichnete klassenlabels (im vergleich GTSRB datensatz 43)

\begin{enumerate}
	\item 
	Aus der aufgabenstellung 64x64x3
	\item 
	bilder aus dem GTSRB [quelle] datensatz
	\item Gekürzte Klassen: Aus der analyse geht die Vermutung hervor, dass nur 33 Klassen unterschieden werden, keine 43 wie im orginal datensatz
	\item Softmax-Ausgabefunktion 
	\item Interpolationsfunktion (vllt mit einem Bild in 3 Interpolationsversionen und jeweiligen Score) 
	\item Overfitting bei Trainingsdaten
	\item unzuverlässigkeit bei nicht-Schildern (z.B. OhmLogo)
\end{enumerate}


