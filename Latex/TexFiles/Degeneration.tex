\chapter{Degeneration}
\section{Konzept}
Hier kommt die Idee, der Pseudocode und die Voraussetzungen damit es klappt

\section{Implementierung Remote}
Hier kommt der konkrete Code, das trennen der Alternation-Funktionen und einige sonstige Ideen hin 
\section{Ergebnisse Remote}
Hier kommen ein paar Beispiele und Plots. 

Auch hierhin kommt ein Fazit welche AlternationFunktionen wie gut waren und das die Trainingsbilder nicht geeignet waren

Sehr wichtig sind die benötigten Zeiten.
\section{Implementierung Lokal}
Ich weiß nicht ob das ein extra Punkt ist, aber an sich würde ich hier das Model bei uns kurz vorstellen

\section{Anpassung und Verbesserung Lokal}
Hier kommen zunächst so Dinge wie "wait" rauszunehmen aber GPU-Acceleration würde ich auch hernehmen. 
\subsection{Batch-Degeneration}
Ist noch ein To-Do: Degenerieren von 100 Bildern, Wahl des "besten". Mach ich noch.
\subsection{Parallel-Degeneration}
Nur Ansatz: Hat nicht geklappt das zu basteln weil numpy Arrays echt nervig sind bei Parallelverarbeitung. 

Vorstellen kann man das allerdings kurz. Konflikt mit GPUAcceleration. 